\documentclass{article}
\usepackage{xcolor}
\usepackage{natbib}
\usepackage{authblk}
\usepackage{amsmath}
\usepackage{amssymb}
\usepackage{graphicx}
\usepackage{pdflscape}
\usepackage{hyperref}
\usepackage{booktabs}
\usepackage{geometry}
\usepackage{longtable}

\newcommand*\mean[1]{\overline{#1}}
\newcommand{\var}{\text{var}}
\newcommand{\Var}[1]{\text{Var}(#1)}
\newcommand{\cov}{\text{cov}}
\newcommand{\cor}{\text{cor}}
\newcommand{\Rp}{\text{Re}}
\newcommand{\E}{\text{E}}
\newcommand{\Mat}[1]{\text{\textbf{#1}}}
\newcommand{\ltsgr}{\text{ltsgr}}
\renewcommand{\thetable}{S\arabic{table}}
\renewcommand{\thefigure}{S\arabic{figure}}
\renewcommand{\baselinestretch}{1.5}
\newcommand{\expit}{\text{expit}}
\newcommand{\logit}{\text{logit}}

\title{Manual for the R package xsdmMle}

\author[a]{\'{A}ngel Luis Robles Fern\'{a}dez}
\author[b]{Emilio Berti}
\author[a]{Daniel C. Reuman}

\affil[a]{Department of Ecology and Evolutionary Biology and Center for Ecological Research, University of Kansas, Higuchi Hall, Lawrence, KS, 66047, USA\vspace{0.5em}}
\affil[b]{Institute of Biodiversity, Friedrich-Schiller University Jena, F\:{u}rstengraben 1, Jena, 07743,
Germany}

\begin{document}
\maketitle
\tableofcontents

Note: right now I just took all the math I wrote and dumped it in here. There's a big
difference between the math and the manual, but I have not really though nearly
enough about how the manual will be organized to make this more like a manual that
will be useful for people starting to use the package for the first time. So right
now it bears little resemblance to what it will eventually look like. ALL the math 
will be included, though, in edited form, and most in the appendices or later 
``advanced'' chapters so I dumped it all in here for now. 

\section{The model}

\subsection{Overview of model structure}

We assume a population, $p_t$, of a species in a location follows the model 
$p_{t+1} = \lambda_t p_t$, where $\lambda_t$ is a net population growth rate.
Given an environmental time series $\vec{e}_t$ for the location, the model estimates 
the habitat suitability of the location for the species in three steps. First,
a ``growth-environment function" model component postulates that 
$\lambda_t$ is a function of $\vec{e}_t$ and a first set 
of model parameters, $\theta_1$. 
Second, the long-term stochastic growth rate 
(ltsgr) for the location is computed as 
\begin{equation}
\text{ltsgr} = \mean{\log(\lambda_t)},
\end{equation}
where the overbar represents the average through time. All 
logarithms in this paper are natural logarithms. Third, the habitat suitability of
the location is assumed to be a sigmoid function of the ltsgr, with the specific
shape of the sigmoid controlled by a second set of model parameters, $\theta_2$.
This component of the model is called the ``detection link''.
For model confrontation with species occurrence data and (pseudo-)absences, 
habitat suitability is assumed to represent a probability of occurrence.

\subsection{The growth-environment function}\label{sect:defns}

Let $n$ be a positive integer representing the number of environmental variables which
will influence population growth, i.e., $n$ is the dimension of $\vec{e}_t$. Let $\vec{\mu}=(\mu_1,\ldots,\mu_n)$ be an $n$-vector of unconstrained
real values representing the optimal values for growth of each environmental variable. 
Let $\vec{\sigma}_L=(\sigma_{L,1},\ldots,\sigma_{L,n})$ and $\vec{\sigma}_R=(\sigma_{R,1},\ldots,\sigma_{R,n})$ be $n$-vectors of strictly positive real
values representing widths of the growth-environment function to the left and to 
the right, respectively, with respect to the corresponding environmental variable.
Let $O$ be an $n \times n$ orthogonal matrix (i.e., $OO^\tau = I$, where the $\tau$ represents matrix transpose). Let $\lambda_{\text{max}}$ represent 
population growth under optimal conditions, i.e., the maximum possible annual net
growth rate. The growth-environment function is defined as
\begin{equation}
\lambda_t = \lambda_{\text{max}} \prod_{i=1}^n \exp\left( -\frac{1}{2} \left( \frac{u_{t,i}}{\sigma_i(u_{t,i})}  \right)^2 \right),
\label{eq:gefunc}
\end{equation}
where $\vec{u}_t = O^{-1}(\vec{e}_t-\vec{\mu})$ and $\sigma_i(u_{t,i}) = \sigma_{L,i}$ if 
$u_{t,i} \leq 0$ and $\sigma_i(u_{t,i}) = \sigma_{R,i}$ if $u_{t,i} > 0$. 
Fig. X (to be added) shows examples of functions that can be
obtained from (\ref{eq:gefunc}) in one and two dimensions. 
%Add a figure showing examples in 1d and 2d and cite the figure here

We now elaborate the reasoning behind (\ref{eq:gefunc}). In the univariate case,
(\ref{eq:gefunc}) reduces to
\begin{equation}
\lambda_t= \lambda_{\text{max}}\exp \left( -\frac{1}{2} \left( \frac{e_t - \mu}{\sigma(e_t-\mu)} \right) ^2 \right),
\end{equation}
where $\sigma(e_t-\mu) = \sigma_L$ if $e_t \leq \mu$, and $\sigma(e_t-\mu) = \sigma_R$ if $e_t > \mu$. This function is proportional to an asymmetric generalization of the
probability density function (pdf) of a normal distribution.
This form is convenient because annual net growth rate should decline monotonically to zero 
for extreme values of $e_t$, though it may do so asymmetrically. 
For similar reasons, the multivariate growth-environment function is an asymmetric 
generalization of a functional form used in the pdf of the multivariate normal distribution,
\begin{equation}
\exp\left( -\frac{1}{2}(\vec{e}_t-\vec{\mu})^\tau \Sigma^{-1} (\vec{e}_t-\vec{\mu}) \right),
\label{eq:multivarnorm}
\end{equation}
where $\Sigma$ is a positive-definite covariance matrix. 
To define an asymmetric generalization of eq. (\ref{eq:multivarnorm}), we first 
note, by the spectral theorem, 
that we can write $\Sigma= O D O^{-1}$ for $O$ an orthogonal matrix
and $D$ a diagonal matrix with positive diagonal entries. The columns of $O$ are
the eigenvectors of $\Sigma$ and the diagonal entries of $D$ are the corresponding 
eigenvalues. 
Then,
\begin{align}
\exp\left(-\frac{1}{2}(\vec{e}_t - \vec{\mu})^\tau \Sigma^{-1} (\vec{e}_t - \vec{\mu})\right) &= 
\exp\left(-\frac{1}{2}(\vec{e}_t - \vec{\mu})^\tau 
O D^{-1} O^{-1}  (\vec{e}_t - \vec{\mu})\right) \\ 
&= \exp\left(-\frac{1}{2}[O^{-1} (\vec{e}_t - \vec{\mu})]^\tau D^{-1} [O^{-1} (\vec{e}_t - \vec{\mu})]\right) \\
&= \exp\left(-\frac{1}{2}\vec{u}_t^\tau D^{-1} \vec{u}_t\right),
\end{align}
where $\vec{u}_t = O^{-1} (\vec{e}_t - \vec{\mu})$. Letting $\sigma_i$ be the 
square root of the $i$th diagonal entry of $D$, this is 
\begin{equation}
\exp\left(-\frac{1}{2}\sum_{i=1}^n \left( \frac{u_{t,i}}{\sigma_i} \right)^2\right)=
\prod_{i=1}^n \exp\left(-\frac{1}{2} \left( \frac{u_{t,i}}{\sigma_i} \right)^2\right).
\end{equation}
Introducing asymmetry, this generalizes straightforwardly to 
\begin{equation}
\prod_{i=1}^n \exp\left(-\frac{1}{2} \left( \frac{u_{t,i}}{\sigma_i(u_{t,i})} \right)^2\right).
\end{equation}
where $\sigma_i(u_{t,i}) = \sigma_{L,i}$ if 
$u_{t,i} \leq 0$ and $\sigma_i(u_{t,i}) = \sigma_{R,i}$ if $u_{t,i} > 0$.
Multiplying by $\lambda_{\text{max}}$ then gives (\ref{eq:gefunc}).

\subsection{Getting a sense for what growth-environment functions look like}

I currently have a figure cited above which does not exist yet. Maybe it is enough to 
have such a figure but maybe you actually need a whole section. Adding it here to
be filled in later.

\subsection{The detection link}

The detection link is 
\begin{equation}
\frac{p_d}{1+exp(-b(\text{ltsgr}-c))},
\end{equation}
where $0<p_d \leq 1$, $b>0$, and $c$ is an unconstrained real number.

\section{The likelihood function}\label{sect:like}

The long-term stochastic growth rate (ltsgr) is 
\begin{align}
\text{ltsgr} &= \mean{\log(\lambda_t)} \\
&= \log(\lambda_{\text{max}}) - \frac{1}{2}\sum_{i=1}^n \mean{\left( \frac{u_{t,i}}{\sigma_i(u_{t,i})} \right)^2}. 
\end{align}
Plugging this into the detection link gives a probability of detection for 
the location of 
\begin{align}
P(X=1) &=
\frac{p_d}{1+\exp\left( -b(\text{ltsgr}-c) \right)} \\
&= \frac{p_d}{1+\exp\left(\frac{b}{2}\sum_{i=1}^n \mean{\left( \frac{u_{t,i}}{\sigma_i(u_{t,i})} \right)^2}  +b(c-\log(\lambda_{\text{max}}))\right)} \\
&= \frac{p_d}{1+\exp\left(\frac{1}{2}\sum_{i=1}^n \mean{\left( \frac{u_{t,i}}{\sigma_i(u_{t,i})/\sqrt{b}} \right)^2}  +b(c-\log(\lambda_{\text{max}}))\right)} \label{eq:structnonident}\\
&= \frac{p_d}{1+\exp\left(\frac{1}{2}\sum_{i=1}^n \mean{\left( \frac{u_{t,i}}{\tilde{\sigma}_i(u_{t,i})} \right)^2}  +\tilde{c}\right)},\label{eq:probdetect}
\end{align}
where $\tilde{c} = b(c-\log(\lambda_{\text{max}}))$ and 
$\tilde{\sigma}_i(u_{t,i}) = \tilde{\sigma}_{L,i} \equiv \sigma_{L,i}/\sqrt{b}$
for $u_{t,i} \leq 0$ and $\tilde{\sigma}_i(u_{t,i}) = \tilde{\sigma}_{R,i} \equiv \sigma_{R,i}/\sqrt{b}$
for $u_{t,i} > 0$. This parameter reduction is to eliminate the structural non-identifiability
in the model, visible in (\ref{eq:structnonident}). As a result of the parameter reduction, 
the probability of detection (and, subsequently, the 
likelihood, defined below) is a function of $\vec{e}_t$, $\vec{\mu}$, $O$, $\tilde{\vec{\sigma}}_L$,
$\tilde{\vec{\sigma}}_R$, $\tilde{c}$, and $p_d$. If $P_i$ is the probability of 
detection, just defined, for location $i$, then the likelihood is 
$\prod_i P_i \prod_j (1-P_j)$, where the first product is over all detection locations
and the second is over all locations of absence or pseudo-absence.

\section{Maximizing the likelihood}

\subsection{The ``math scale'' and the ``biological scale''}

Some of the parameters the likelihood function depends on
($\vec{\mu}$, $O$, $\tilde{\vec{\sigma}}_L$,
$\tilde{\vec{\sigma}}_R$, $\tilde{c}$, and $p_d$; see section \ref{sect:like})
are constrained. In order to be able to optimize the likelihood function working on
an unconstrained space, we here define such a space, and we 
describe transformations that convert parameters between the unconstrained and
constrained spaces. 

Parameters in the unconstrained space are henceforth called
``math-scale'' parameters, and constrained parameters defined in 
section \ref{sect:like} are called ``biological-scale'' parameters. The transformations
between the two spaces act separately on each parameter. When a 
distinction is needed, we denote biological-scale parameters with a superscript
$(b)$, and math-scale parameters with a superscript $(m)$, so the biological-scale
parameters are denoted $\vec{\mu}^{(b)}$, $O^{(b)}$, $\tilde{\vec{\sigma}}_L^{(b)}$,
$\tilde{\vec{\sigma}}_R^{(b)}$, $\tilde{c}^{(b)}$, and $p_d^{(b)}$; and the math-scale
parameters are denoted $\vec{\mu}^{(m)}$, $O^{(m)}$, $\tilde{\vec{\sigma}}_L^{(m)}$,
$\tilde{\vec{\sigma}}_R^{(m)}$, $\tilde{c}^{(m)}$, and $p_d^{(m)}$.
The transformations for all the parameters except $O$ are
\begin{align}
\vec{\mu}^{(b)} &= \vec{\mu}^{(m)} \label{eq:mutrans}\\
\tilde{\vec{\sigma}}_L^{(b)} &= \exp(\tilde{\vec{\sigma}}_L^{(m)}) \label{eq:sigLtrans}\\
\tilde{\vec{\sigma}}_R^{(b)} &= \exp(\tilde{\vec{\sigma}}_R^{(m)}) \label{eq:sigRtrans}\\
\tilde{c}^{(b)} &= \tilde{c}^{(m)} \label{eq:ctrans}\\
p_d^{(b)} &= \expit(p_d^{(m)}), \label{eq:pdtrans}
\end{align}
where the $\exp$ of a vector is interpreted to be the vector resulting from
$\exp$ transforming each component, and where $\expit$ is the standard logistic sigmoid
function $1/(1+\exp(-x))$, which is the inverse of the $\logit$ function.
The parameter $O^{(m)}$ is interpreted to be an unconstrained Euclidean vector
of dimension $\frac{n^2-n}{2}$, and $O^{(b)}$ is obtained from $O^{(m)}$
by using $O^{(m)}$ to form a skew-symmetric matrix of dimensions $n \times n$, 
and then applying the matrix exponential map to that skew-symmetric matrix. It is known that
the matrix exponential of a skew-symmetric matrix is an orthogonal matrix.
The unconstrained space of parameters (the space of math-scale parameters) 
is thus a Euclidean space of dimensions $3n+\frac{n^2-n}{2}+2$,
where: the $3n$ term in this expression comes from the parameters $\vec{\mu}^{(m)}$,
$\tilde{\vec{\sigma}}_L^{(m)}$, and $\tilde{\vec{\sigma}}_R^{(m)}$;
the $2$ in this expression comes from the parameters $\tilde{c}^{(m)}$ and
$p_d^{(m)}$; and the term $\frac{n^2-n}{2}$ in the expression comes from
the parameters $O^{(m)}$.

The inverse transformation, from the space of biological-scale parameters to the
space of math-scale parameters, is only a partial inverse. The maps
(\ref{eq:mutrans})-{\ref{eq:pdtrans}) are readily invertible, but the matrix
exponential map from the space of skew-symmetric matrices to the space of
orthogonal matrices is many-to-one and the range comprises the special orthogonal 
matrices. We use as a partial inverse
the principle branch of the matrix logarithm. 

\subsection{Initial conditions, and optimizations}\label{sect:inits}

We expressed the log likelihood as a function of math-scale parameters and 
optimized it with standard numeric optimizers. This required the specification of 
initial conditions from which to start the optimizations. Multiple initial conditions
were typically used in order to increase the chances of finding the global
maximum of the log likelihood function. We here describe how reasonable parameter initial 
conditions were selected. (To be filled in.)
%fill

\subsection{Initial screening for successful optimization}\label{sect:screening}

After carrying out multiple optimizations of the log likelihood 
from distinct initial prameter values
as described in section \ref{sect:inits}, results of the optimizations
were compared to help infer characteristics
of the objective surface and to help judge the likelihood that a global optimum was found.
For instance, if 1) the best optimization result converged according to the optimizer's 
assessment of convergence; and 2) multiple optimizations resulted in maximized log likelihood values
which were within a small increment, $\epsilon$, of the overall best result;
and 3) if all these results were tightly clustered in parameter space but the initial
conditions that led to them were spread out; that suggests the log likehood surface has
a dominant peak with a large ``basin of attraction'' which is plausibly the global 
maximum.  On the other hand, if 1) the best optimization result did not converge; or 
2)  secondary optimization results had optimized log likelihood values more than 
$\epsilon$ less than the best one, or 3) the best optimization results were not tightly
clustered in parameter space; that suggests one of several log-likelihood pathologies
may have occurred which required additional exploration and possibly model reformulation. 
Pathologies include: no internal maximum of the log-likelihood surface (i.e., the supremum of
the log-likelihood is approached as some parameter asymptotically approaches a boundary
value - see section \ref{sect:profiling_and_boundary} below); and 
multiple local maxima of the of the log-likelihood surface exist, with potentially small 
``basins of attraction'' and 
correspondingly lower probability that the global maximum was dicovered.

Many assessment of optimization success such as those described above 
require comparison of optimized parameters.
Parameter comparison for the XSDM model is complicated by the fact that our parameterizations
described in previous sections are redundant, with different parameters refering to the 
same underlying model and resulting in the same log-likelihood value. We here describe these 
redundancies and how to compensate for them in both juding the likelihood of 
optimization success and in downstream interpretations of opimization results.

If $\Sigma$ is a positive definite covariance matrix and $\Sigma = O D O^{-1}$
is the decomposition introduced in section \ref{sect:defns}, 
the matrices $O$ and $D$ are not determined uniquely by $\Sigma$; but, if the 
diagonal entries of $D$ (i.e., the eigenvalues of $\Sigma$) are distinct, $O$ and $D$ are
determined uniquely up to reordering matrix columns and up to multiplying any of the 
columns of $O$ by $-1$. To see this, let $O_1$ (orthogonal) and $D_1$ (diagonal with
positive diagonal entries) satisfy $\Sigma=O_1 D_1 O_1^{-1}$.
This happens if and only if $\Sigma O_1 = O_1 D_1$, which happens for an orthogonal 
matrix $O_1$ if and only if the columns of $O_1$ are an orthonormal 
eigenbasis for $\Sigma$.  Reordering the columns of $O_1$ and multiplying any
of them by $-1$ results in another matrix, $O_2$, the columns of which are another
orthonormal eigenbasis of $\Sigma$. The columns of $D_1$ have to be reordered
in the same manner, the resulting matrix denoted
$D_2$, and then we have $\Sigma O_2 = O_2 D_2$ and therefore $\Sigma = O_2 D_2 O_2^{-1}$. 
On the other hand, if $\Sigma = O_2 D_2 O_2^{-1}$ for an orthogonal matrix $O_2$ and
a diagonal matrix with positive diagonal entries, $D_2$, then 
$\Sigma O_2 = O_2 D_2$, so $O_2$ is another orthonormal eigenbasis for $\Sigma$
with corresponding eigenvalues along the diagonal of $D_2$. This means $D_2$ is related
to $D_1$ via a re-ordering (the set of eigenvalues of $\Sigma$ is determined by $\Sigma$). 
Under the assumption that the eigenvalues of $\Sigma$
are distinct, the eigenvectors of $\Sigma$ are also determined up to multiplication
by scalars. But, by orthonormality, those scalars must be $\pm 1$.

If $\Sigma= O_1 D_1 O_1^{-1}$ and the eigenvalues of $\Sigma$ are not distinct, then
there is additional redundancy in the parameterization of $\Sigma$ by $O_1$ and $D_1$. 
For each eigenvalue $d_i$ that occurs with
multiplicity, the corresponding eigenvector columns of $O_1$
can be replaced by any other orthonormal basis of the subspace they span.
The resulting matrix $O_2$ will satisfy $\Sigma= O_2 D_1 O_2^{-1}$. 
In essence, the columns of $O$ specify to an eigenspace decomposition
of $n$-dimensional Euclidean space, guaranteed by the spectral theorem to 
exist and to consist of eigenspaces with real, positive eigenvalues.
Though the eigenspace decomposition is unique, its representation by
eigenvectors is non-unique as described.

In a similar manner, the parameterization of the growth-environment function
$\lambda_t(\vec{e}_t,\theta_1)$ by the parameters 
$\theta_1=\{  \vec{\mu}, \vec{\sigma}_L, \vec{\sigma}_R, O, \lambda_{\text{max}} \}$
is redundant, in the sense that different parameter values render the same function
of $\vec{e}_t$. Given any permutation, $\phi$, of $n$ elements, if $\phi$ is
applied both to the columns of $O$ and to the elements of both $\vec{\sigma}_L$ 
and $\vec{\sigma}_R$, the same growth-environment function results. Also, if any 
column, $i$, of $O$ is multiplied by $-1$ and the $i$th elements of $\vec{\sigma}_L$ 
and $\vec{\sigma}_R$ are exchanged, the same growth-environment function results.  
Finally, if there exists an index set $i_1,\ldots,i_l$ for which the parameters
$\sigma_{L,i_1},\ldots,\sigma_{L,i_l}$ and $\sigma_{R,i_1},\ldots,\sigma_{R,i_l}$
are all equal, the corresponding columns of $O$ can be replaced by any other
orthonormal basis of the space they span, and the same growth-environment
function results. 

Now turning to the likelihood function, similar statements are also
true, i.e., different values of the parameters $\vec{\mu}^{(b)}$,
$\tilde{\vec{\sigma}}_L^{(b)}$, $\tilde{\vec{\sigma}}_R^{(b)}$, $\tilde{c}^{(b)}$,
and $p_d^{(b)}$ systematically give the same likelihood. Specifically,
given any permutation, $\phi$, of $n$ elements, if $\phi$ is
applied both to the columns of $O$ and to the elements of both $\tilde{\vec{\sigma}}_L^{(b)}$ 
and $\tilde{\vec{\sigma}}_R^{(b)}$, the same likelihood value results.
Also, if any 
column, $i$, of $O$ is multiplied by $-1$ and the $i$th elements of 
$\tilde{\vec{\sigma}}_L^{(b)}$ 
and $\tilde{\vec{\sigma}}_R^{(b)}$ are exchanged, the same likelihood 
value results.
Finally, if there exists an index set $i_1,\ldots,i_l$ for which the parameters
$\tilde{\sigma}_{L,i_1}^{(b)},\ldots,\tilde{\sigma}_{L,i_l}^{(b)}$ and 
$\tilde{\sigma}_{R,i_1}^{(b)},\ldots,\tilde{\sigma}_{R,i_l}^{(b)}$
are all equal, the corresponding columns of $O$ can be replaced by any other
orthonormal basis of the subspace they span, and the same likelihood value results. 

The map from the space of math-scale parameters to the space of 
biological-scale parameters is not bijective, introducing additional complexities
that must be managed. 
First, the matrix exponential map from the space of skew-symmetric matrices to the
space of orthogonal matrices is not surjective because the range of the map
is the special orthogonal matrices, i.e., those with determinant $1$, whereas 
orthogonal matrices can have determinant $1$ or $-1$. This makes no difference,
however, because, given any orthogonal matrix with determinant $-1$, one can
multiply any single column by $-1$ or switch two columns, changing the determinant to
$1$. As discussed above, this will not alter the resulting growth-environment function
or likelihood value; so our parameterization covers all necessary $O^{(b)}$ values.
Equations (\ref{eq:mutrans})-(\ref{eq:pdtrans}) define bijective maps, except 
$p_d^{(b)}=1$ is not in the range of the $p_d$ map; this bounadry case is dealt with in section
\ref{sect:profiling_and_boundary}), below. 

The matrix exponential map from skew-symmetric matrices to special orthogonal matrices is 
also many-to-one. This does not prevent us
from maximizing the likelihood function on the math scale, but it does mean that
there will be multiple optima on the math scale which have the same likelihood value
and correspond to the same parameters on the biological scale. 

The way we deal with the various forms of redundancy outlined above during the effort,
outlined at the beginning of this section, of assessing whether we successfully
optimized, is as follows. Suppose we ran $N$ log-likelihood
optimizations, and the resulting log-likelihood values are $L_i$, $i=1,\ldots,N$
and best parameters from each optimization are 
$\vec{\mu}^{(m),i}$, $O^{(m),i}$, $\tilde{\vec{\sigma}}_L^{(m),i}$,
$\tilde{\vec{\sigma}}_R^{(m),i}$, $\tilde{c}^{(m),i}$, and $p_d^{(m),i}$,
$i=1,\ldots,N$. We assume the results have been sorted so that $L_1 \geq L_2 \geq 
\ldots, L_N$. We want to assess whether the top several results produced similar
parameters, but we know differing math-scale parameters can represent the same
underlying model. So, prior to comparing parameters from different optimizations,
we convert all parameter results to the biological scale, 
$\vec{\mu}^{(b),i}$, $O^{(b),i}$, $\tilde{\vec{\sigma}}_L^{(b),i}$,
$\tilde{\vec{\sigma}}_R^{(b),i}$, $\tilde{c}^{(b),i}$, and $p_d^{(b),i}$,
$i=1,\ldots,N$. But we know differing biological-scale parameters can also
represent the same underlying model, the redundancy, described above, lying
with the parameters $O^{(b),i}$, $\tilde{\vec{\sigma}}_L^{(b),i}$, and
$\tilde{\vec{\sigma}}_R^{(b),i}$. So we convert these biological-scale parameters
into ``canonical form'' as follows. We first multiply columns of $O^{(b),i}$
by $-1$, as necessary, to ensure that the top element of each column is non-negative.
We meanwhile switch corresponding elements of $\tilde{\vec{\sigma}}_L^{(b),i}$ and
$\tilde{\vec{\sigma}}_R^{(b),i}$, to ensure that our alterations do not alter the
underlying likelihood. We then re-order the columns to put them in dictionary order.
We meanwhile reorder the entries of  $\tilde{\vec{\sigma}}_L^{(b),i}$ and
$\tilde{\vec{\sigma}}_R^{(b),i}$ using the same permutation, again 
to ensure that our alterations do not alter the
underlying likelihood. The resulting parameters for $i=1,\ldots,N$ can then 
be directly compared.

\section{Profiling and boundary models}\label{sect:profiling_and_boundary}

\subsection{Profiling} 

To be filled in. We profile on the math scale.

\subsection{The need for boundary models}

Ocassionally profiles are non-peaked, e.g., the profile appears to asymmptotically
approach some supremum log-likelihood value from below as the profile parameter takes more
and more extreme values to the right or to the left. Profiles can also be weakly peaked, 
i.e., exhibiting a maximum but then approaching a log-likelihood asymptote, from above,
as the profile parameter takes extreme values to the left or right; and the asymptote
log-likelihood value is not much less than the maximum.
Examples of these cases are in Fig. X.
%DAN: Need to make these examples and make a figure. Make sure it has th significance threshold.
The first of these cases indicates that the likelihood does not have an ``internal maximum"
in the space of math-scale parameters. For instance, it is not uncommon for profiles 
with respect to the parameter $p_d^{(m)} = \logit(p_d^{(b)})$ to be unpeaked, 
asymptotically approaching some log-likelihood value as $p_d^{(m)} \rightarrow \infty$
(Fig. Xa). %DAN: Make sure one of your panels shows this.
This indicates that the value $p_d^{(b)}=1$ would produce a model that is as 
well or better supported by data than a model with $p_d^{(b)}<1$, though our fitting
and profiling procedures do not consider $p_d^{(b)}=1$ or other boundary cases because
of the parameterizations we use ($1$ is not in the range of the $\logit$ map). 
We remind the reader that maximum likelihood
results and profiles are only valid, strictly speaking, when the likelihood has a maximum
and when the likelihood surface
in the vacinity of the maximum resembles a parabaloid, in particular meaning
profiles are peaked and approximately resemble inverted parabolas. Cases such as those
described above indicate that a simpler, ``boundary model'' (e.g., one with
$p_d^{(b)}$ fixed at $1$) should be fitted with occurrence data. 
We next formulate boundary models.

\subsection{The boundary models and related models}

The first boundary model is straightforward. Because the log-likelihood, which is 
based on (\ref{eq:probdetect}), is well defined for $p_d^{(b)}=1$, this values can 
be fixed. The likelihood then depends on the other parameters, which can be considered
in the math or biological scale (the math scale is used for maximizing and profiling). 
We do not consider the boundary case $p_d^{(b)}=0$ because the likelihood will be 
zero for that case if there are any detections of the species. 

We also consider the set of boundary models for which, for any given $i$, one (but not 
both) of $\tilde{\sigma}_{L,i}$ and
$\tilde{\sigma}_{R,i}$ is infinite, so that $u_{t,i}/\tilde{\sigma}_{L,i}$
or $u_{t,i}/\tilde{\sigma}_{R,i}$ is interpreted to be $0$ in (\ref{eq:probdetect}).
This corresponds to a case where population growth is insensitive to extreme
values of an environmental variable in one extreme but not the other.
We do not here consider $\tilde{\sigma}_{R,i}=\tilde{\sigma}_{R,i}=\infty$ because that
is equivalent to a model where population growth is influenced by $n-1$ 
environmental variables.  Equation (\ref{eq:probdetect}) is still defined as 
a function of the other parameters, and the likelihood can be computed, and maximized
on the math scale, and profiled. There are $3^n-1$ such boundary models, because for
each index $i=1,\ldots,n$, one can set to $\infty$ the parameter $\tilde{\sigma}_{L,i}$ or
$\tilde{\sigma}_{R,i}$ or neither, but not both (this gives $3^n$); but the model
where none of these is set to $\infty$ is the original model, not a boundary model
(making the final count $3^n-1$). Typically a boundary model of this type is
only fitted if maximization or profiling of the main model indicates it is
warranted, for instance if the profile of one of the parameters $\tilde{\sigma}_{L,i}$ and
$\tilde{\sigma}_{R,i}$ asymptotes as the parameters goes to $\infty$ instead of having 
a maximum at a finite value. For this paragraph, it does not matter whether we
use the math- or biological-scale parameters, so we have not used either superscript
$(m)$ or the superscript $(b)$.

We consider one additional, optional type of simplified model.
If the maximum-likelihood estimates 
$\hat{\tilde{\sigma}}_{L,1},\ldots,\hat{\tilde{\sigma}}_{L,n}$ and 
$\hat{\tilde{\sigma}}_{R,1},\ldots,\hat{\tilde{\sigma}}_{R,n}$
are all equal or close to equal, the maximum-likelihood estimate
$\hat{O}^{(b)}$ will not
be precisely determined by the data. For instance, in the $n=2$ case, if
maximum-likelihood estimates $\hat{\tilde{\sigma}}_{L,1}$, $\hat{\tilde{\sigma}}_{L,2}$,  
$\hat{\tilde{\sigma}}_{R,1}$, and $\hat{\tilde{\sigma}}_{R,2}$ are all equal, then the
level curves of the growth environment function are circles, and all possible
$O^{(b)}$ give the same growth environment function and therefore the same 
likelihood. If the maximum-likelihood estimates
of $\hat{\tilde{\sigma}}_{L,1}$, $\hat{\tilde{\sigma}}_{L,2}$,  
$\hat{\tilde{\sigma}}_{R,1}$, and $\hat{\tilde{\sigma}}_{R,2}$ are very similar but not equal,
then the level curves of the growth environment function are nearly circles,
and different $O^{(b)}$ values result in different but very similar growth-environment
functions which should produce similar likelihood values. In such a case, 
likelihood profiles of the parameters in $O^{(m)}$ should be flat or close to 
flat, rendering it statistically impossible to determine specific values of 
$O^{(m)}$ with any certainty. We hasten to add, however, that this makes little
practical difference because all these alternative values of $O^{(m)}$ give similar
growth-environment functions and therefore similar models, which will result
in similar habitat suitability maps and range maps for the species.
Returning to the case of general $n$, we consider the simplified model where 
all the parameters $\tilde{\sigma}_{L,1},\ldots,\tilde{\sigma}_{L,n}$ and 
$\tilde{\sigma}_{R,1},\ldots,\tilde{\sigma}_{R,n}$
are constrained to be equal, replacing $2n$ free parameters by one free 
parameter; and where the matrix parameter $O^{(b)}$ is fixed at the identity matrix.
This greatly simplified model should typically be easier to fit,
and can be compared via AIC (or another criterion) to the original model.
Cases where there exists an index set $i_1,\ldots,i_l$ with $1<l<n$ for which the 
maximum-likelihood estimates 
$\hat{\tilde{\sigma}}_{L,i_1},\ldots,\hat{\tilde{\sigma}}_{L,i_l}$ and 
$\hat{\tilde{\sigma}}_{R,i_1},\ldots,\hat{\tilde{\sigma}}_{R,i_l}$
for the original model are all equal or close to equal can also suggest that
reduced models may be worth considering, but
we leave such considerations for future work. 

\end{document}